\documentclass[twoside,final,12pt]{extreport}
\usepackage[utf8x]{inputenc}
\usepackage[russianb]{babel}
\usepackage{vmargin}
\setpapersize{A4}
\setmarginsrb{2.5cm}{2cm}{1.5cm}{2cm}{0pt}{0mm}{0pt}{13mm}
%\usepackage{indentfirst}
\sloppy
\begin{document}

\chapter {ПРОТИВОАРИТМИЧЕСКИЕ СРЕДСТВА}
\section {КЛАССИФИКАЦИЯ}
{\bfseries КЛАСС 1. БЛОКАТОРЫ НАТРИЕВЫХ КАНАЛОВ}
\begin{table}[t]
\begin{tabular}{|p{0.33\textwidth}|p{0.33\textwidth}|p{0.33\textwidth}|}
	\hline \ 1A & 1B & 1C \\ \hline
	Умеренное замедление фазы 0 деполяризации и проведения, удлинение реполяризации &
	Минимальное замедление фазы 0 деполяризации и проведения, укорочение реполяризации &
	Выраженное замедление фазы 0 и проведения \\ \hline
	Новокаинамид
	Хинидин
	Дизопирамид=Ритмодан &
	Лидокаин 
	Мексилетин 
	Дифенин &
	Пропафенон
	Морацизин \\ \hline
\end{tabular}
\caption{КЛАСС 1. БЛОКАТОРЫ НАТРИЕВЫХ КАНАЛОВ}
\label{Na-block}
\end{table}


{\bfseries КЛАСС 2. БЕТА-БЛОКАТОРЫ}


{\bfseries КЛАСС 3. БЛОКАТОРЫ КАЛИЕВЫХ КАНАЛОВ}


{\bfseries КЛАСС 4. БЛОКАТОРЫ КАЛЬЦИЕВЫХ КАНАЛОВ}


{\bfseries ВСПОМОГАТЕЛЬНЫЕ ПРЕПАРАТЫ}
\end{document}
